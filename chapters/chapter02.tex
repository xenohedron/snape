% A Difference in the Family: The Snape Chronicles (Rannaro)

\chapter{The Perfect Baby}

\subsection{Saturday, January 9, 1960 \\ (halfway between first quarter and full moon)}

Exactly nine months later, at one forty-six in the afternoon on Saturday, the ninth of January, 1960, Eileen Snape gave birth to a tiny baby boy in that small mill town about fifteen minutes drive from Colne in the Pendle district of Lancashire. There was no doctor present, and no licensed midwife either, the birth being assisted by the new baby's two grandmothers. If the child had been born in a hospital, he would have been taken from the room and placed in a crib in an infant ward, swaddled and removed from real human contact except for the short time each day when he would be allowed to be held by his mother. At home, however, he was washed and placed immediately in Eileen's arms, so that the first thing he ever really noticed in his life, blurred and unfocused though they were, were his mother's eyes.

His father Toby was so overcome by the proxy pain he felt for his wife's travail, that he spent the whole morning in the local pub, together with his own father, Edward Snape, and only returned home, boisterous and joyful, after he was assured he was the father of a reasonably healthy son. Wensley spent the entire time in the sitting room or bringing water and towels to the women. He couldn't go into the upper room itself, for that was women's domain.

"Look a' that!" Toby chirped when he saw his son for the first time. "I told ye no son o' mine 'd be born bald! Chip off the ol' block, 'e is. Chip off the ol' block. Thought all babies 'ad blue eyes, though," for the newest member of the family had wisps of dusky hair already, and his eyes were so dark as to look black.

"You can't ever tell a baby's eye color when he's born," admonished Nora Snape, Toby's mother. "Give him a couple of months and they'll change. Ned, don't breathe the smell of whiskey in the boy's face."

"Maybe," said Constantina, looking at the child thoughtfully and still talking about his eyes. "We'll have to wait and see."

Wensley reached out a tentative little finger and touched it to the baby's palm. The tiny fist clutched it with surprising firmness. Constantina sniffed at the expression on the older man's face. "All newborns hold on tightly," she said. "That grip 'll weaken soon."

"You think so?" said Wensley wistfully. "I was kind of hoping he might be a strong 'un."

"He may still be, but not because of the way he's hanging on right now. Everybody out, now. You've all seen him, and 'Leen needs some rest." The three men went downstairs to plan the boy's future and toast him, his mum, and everyone else they could think of, while upstairs the older women began to prepare mother and baby for the first feeding. There was some tension between them, since Constantina's ideas of what was necessary did not coincide with Nora's. In general, the witch's will prevailed.

The boy was not immediately named. Eileen, true to her education, wanted him to be Septimius Severus, but Tobias wasn't going to have any Septimius in his family, by God, and insisted the baby be named for his grandfathers, either Edward Richard, with Tobias's father first, or Richard Edward, giving pride of place to Eileen's. The eventual compromise was Richard Severus, Toby allowing the Richard since Richard Prince had died some years before, and Toby had never gotten along well with his own father anyway. Not unless they were in a pub.

With the logic inherent in all families who spend an inordinate time choosing names, the little boy was never called any variant of Richard, everybody for some reason settling on the nickname Russ. The neighbors, in fact, labored all his life under the misconception that his name was Russell.

Russ Snape was, from the day of his birth, a changeling child. His father put it down to superior intelligence.

"Ain't he the smart one, though, 'Leen. He knows when I come home from work I need it nice and quiet. He don't never bother, does he? Quiet as a parson. Look at him pushing his head up to look around. Won't nobody never put nothing over on him."

Eileen watched the development of her son with tigerish pride as he stretched and kicked and explored his little world and, when put on his stomach, pushed himself up to watch her. She didn't tell Toby that the child never cried. It made things easier if Toby felt that was a baby gift just for him. She also didn't tell Toby that the sharp, quick, black eyes lit up and sparkled more for her than for his father. There was no reason reminding her husband that children always have a closer relationship with their mothers.

Eileen's own mother had other things to say when Eileen and her son came visiting.

"What do you mean, he doesn't cry? All babies cry sometimes. When they're hungry, or tired, or they need their nappies changed. He must cry sometimes."

"No, Mum. He doesn't cry. He never has. He has other ways to tell me he needs something." She leaned over the infant. "Russ knows how to tell me things. See? Now he wants me to pick him up and hold him." She lifted the child from Constantina's sofa into her arms. "You want to walk around the room and look at things, don' cha, Russ?"

Constantina, however, had a new worried look on her face. "Eileen, are you reading that baby?" She stood behind her daughter and peered into the great black eyes in the tiny face. "He's closed to me. Not that I could ever read anyone anyway, but I didn't think you could either. No wonder he doesn't cry, if he can just look at you and you know what he wants."

Toby, meanwhile, rapidly became less enchanted, more ambivalent about the child.

"Didn't know as a baby'd cost so much money," he told Eileen. "Bottles and nappies. And can't you just leave 'm a bit to come sit by me? He don't need all your attention. A woman's got to take care of her man."

\subsection{Friday, November 4, 1960 (one day before the full moon)}

There was a little group of women that got together of an afternoon while their husbands worked in the mill. Kate Hanson was widowed by a fire in the cotton shed six years earlier and had no children, but the insurance settlement had allowed her to keep her house, and her skill at needlework—plus taking in boarders—kept her independent. Her younger sister was Polly Heseltine, whose third child, a daughter named Peggy, was two months older than Russ. Other women in the group—they were five altogether—also had small children of preschool age. When they got together, in whosoever house, the children played while the mothers gossiped over a cuppa in the kitchen.

"Peggy 'ad a new word this week," Polly told the group jammed into the Snape's tiny kitchen, clearly proud of her little girl. "Just a year old now, and she says `water,' and `bye-bye,' and just Tuesday she said `pram' as clear as can be." (What Peggy had said was `pam,' but her meaning had been clear.)

"They're such fun t' watch at this age," chimed in Sarah Catlow. "My Bobby's askin' for biscuits and milk. He said `mama' when he was eight months, y' know. How's Russy doing, 'Leen? He'd be getting close."

Eileen poured more tea for Polly. "He ain't started talking yet. Still too young. He'll be walking soon on his own, though." She glanced through the kitchen doorway to where little Russ teetered on his newfound legs, clinging to the sofa. He got around rather quickly now, moving from piece of furniture to piece of furniture, and she had to keep a constant eye on him.

Sarah smiled at the tiny boy, so much frailer than her own sturdy children. "You'll get something soon, 'Leen. His babbling' ll be words 'fore y' know it."

"He doesn't babble either," said Eileen. "He's a quiet one."

"My cousin Jane's girl never babbled," said Edith Phillips, whose son Neil was now using one of Russ's blocks to pound Russ's toes. "They kept a pacifier in 'er mouth t' keep her quiet, and she never said a word 'til she was near three years old, then started talking like a bleeding solicitor. You never can tell."

"Neil," said Eileen from her chair in the kitchen, "don't hit Russ. Neil{\el}"

Neil suddenly let out a howl of pain and rage and sat down plop in the middle of the sitting room. He continued to scream as his mother ran in from the kitchen to pick him up and cuddle him. "What did y' do, love?" Edith crooned to him. "Did you hit your fingers with the old block? That's what naughty little boys get when they hit other people with blocks, y' know. They hit their own fingers." To the other women's expressions of solicitude she replied matter-of-factly, "I don't think he's hurt. Just one of those things, y' know."

Little Russ swayed insecurely where he clung to the sofa, his black eyes intent on the squalling Neil. He hadn't reacted either to the attack with the block or to Neil's tantrum. Eileen kept an eye on him for the rest of the afternoon, but nothing else happened.

The next night was Bonfire Night. Toby loved Bonfire Night because it was his birthday, and he'd grown up with the idea that the fires and the fireworks were for him. He was thirty now, three years older than Eileen, but that put no damper on his enjoyment of the evening.

In fact, Toby spent a good part of the afternoon looking for serviceable pieces of combustible junk to put into the front area, then as dusk gathered he waited in the darkened sitting room, peering through the closed curtains. A group of teenagers came prowling with the first stars, spied the junk, and lifted it carefully and quietly over the low area wall. Toby let them get a ways down the street before he came out yelling at them for thieves. The boys jeered and threw a couple of small stones, and Toby chased them to the town center. It was all in good fun.

Then Toby returned home to collect Eileen and baby Russ, and together they went to watch the bonfire and the burning of the Guy. Eileen made the traditional black treacle cake called parkin, and Toby brought potatoes wrapped in foil to cook in the fire. They `oh'ed and `ah'ed at the fireworks while little Russ watched everything with wide, intent eyes, then went home to feast on potatoes and parkin.

"You got a present for me, 'Leen?" Toby grinned across the lamplit sitting room as the clock ticked past nine.

"I gave you your present," said Eileen. It was a warm winter sweater she'd knitted.

"I 'ad a different present in mind," Toby leered, "seeing as it's m' birthday and all," but Eileen wasn't watching him.

"Shh. Look, Toby. Look at Russ."

Toby looked. Little Russ had clambered to his feet with the help of the sofa. Now, oblivious to the attention he was getting, he released his grip and, holding nothing, staggered toward the front door.

"{\el} two, three, four, five, six, seven, eight{\el}" Eileen stopped counting as Russ lost his balance and sat suddenly on the sitting room floor. He made no sound. "There y' go, Toby! There's your present!" Eileen cried as Toby hugged first her and then the little boy who had him beaming with pride. "Your son's walking!"

\subsection{Saturday, December 24, 1960 (ten hours before the first quarter)}

For some reason passing human understanding, Toby decided to have the whole family over for his, Eileen's, and Russ's first Christmas as a family. Everyone would stay the night, and wake up together for Christmas breakfast and the opening of presents. Since the house was small, they had to split up Toby's parents, for Nora and Constantina had to share the second, smaller bedroom while Wensley and Edward bunked in the sitting room. Ordinarily Toby and Eileen might have given up their bedroom to his parents, but they had Russ's crib there and felt that with all the noise and disruption, the baby should at least be able to sleep in a place that was familiar.

All had arrived by four o'clock Christmas Eve, Wensley, Nora, and Edward in the latter's car with a Christmas tree tied to the top. Nora guarded boxes of fragile ornaments that had been in the family for ages and had graced every one of Toby's Christmases. "Thought it was best for you to have these," she told her son, "now that you're the one with the child."

The men stayed in the sitting room putting up the tree while the women busied themselves in the kitchen preparing supper. Nora watched Constantina and Eileen with barely concealed curiosity. When Constantina raised her eyebrows, Nora admitted. "I was just thinking to see some{\el} you know{\el} I never saw you do any."

"What are you expecting," said Constantina a bit huffily, "an ice sculpture?"

"Plum pudding?" replied Nora.

"That's in the oven already," Eileen said. "I got it this afternoon."

"Oh," Nora sighed sadly. "Just like everyone else."

"That," Constantina said, "is a typical muggle attitude. We can't make anything permanent out of thin air. Magic fades, it dissolves. Magical food doesn't nourish. Magical money turns to dross. Magic is for temporary things, like this{\el}" She set a knife to peeling a potato and a whisk to beating eggs for the eggnog. "Silly muggle idea, using magic to make food."

"Mum!" Eileen hissed. She looked nervous.

"We're inside a witch's house doing simple household tasks," Constantina replied. "Nobody cares. I swear, that school of yours{\el}"

Nora was abashed, though also pleased by the display. "Wensley was always so sure," she said, "but he never could give me a concrete example. Do you think Russy{\el}"

"We don't know," said Constantina. "It's too early to tell."

"Maybe not," Eileen whispered, and the two older women bent closer. "Beginning of November, I was hosting a little group—we all have young children—and one of the boys was hitting Russ with a block, not hard, but hitting and{\el} well all of a sudden he acted like something hit him and pushed him back, but Russ didn't move. I don't know if that was anything, but I've been wondering ever since."

Nora went over to Russ, who was playing in the corner with a toy telephone made of cardboard, holding it to his ear but not imitating talking into it. "Are you gra-gra's little wizard, Russy?" she cooed. "Did you do magic on that naughty boy?" She reached out an arm that had bruises on the wrist. Nora always had bruises on her wrists or arms. It was normal. Russ stared back at her with intent, guarded eyes.

"'Leen," Constantina said suddenly, "you go talk to him. Ask the same question."

Unsure but willing, Eileen took Nora's place. "Russ," she said, "did you do magic on that naughty boy?" and she held the memory of that day in her mind.

Something behind the dark eyes opened then, like doors opening into a lighted hall, sparkling with comprehension and a trace of mischief. Eileen stepped back, puzzlement now on her face.

"He doesn't understand the question," she said, "but he thought 't was funny when Neil fell down."

"That's not just hunger and wet nappies, 'Leen. That's true reading." Constantina pulled a chair away from the table and sat down in it. "All these years," she said, shaking her head. "All these years you had the gift of reading and I never knew. Your own mother, and I never knew. It's because I didn't have the gift. No one in my family had it. There were some in your dad's family, though not him. I suppose that's why I never looked for it in you. My daughter is a reader."

"Does that mean she can read minds?" Nora asked.

"In a way," Constantina said after considering a few seconds whether or not to answer. "She can look in your eyes and know what you're thinking at that moment. Some have it stronger than others."

Nora turned to her daughter-in-law. "What am I thinking now?" she demanded.

Eileen looked. She kept looking. "I don't know," she said at last. "Maybe about a car."

"Close enough," Nora said. "I was thinking about going in the car to get the Christmas tree. There was a car in there. But why," this was addressed to Constantina, "can't she read me better?"

"Maybe it's the baby," Constantina admitted. "Maybe it's just between the mother and her child."

"What's between the mother and her child?" asked a new voice. Wensley Snape was standing in the kitchen doorway.

"'Leen has the gift of reading Russ." Constantina explained. "That's how she always knows what he needs, and probably why he never has to cry to get it. He just lets her see it. It seems now she can read much more than that, though."

"Would he let me see it, too?" Wensley asked. He was an old man, eighty or more, and though he wanted to crouch down at the child's level, he couldn't. Instead he brought a chair and sat next to the boy, leaning forward so that they could make eye contact. "Nothing," he said. "I can't read anything in that little brain."

"That's because you're a muggle, and muggles can't read anyway," Constantina stated flatly, but she'd seen something else. She seen the baby's dark eyes lose the light, as if the door behind them had closed. It opened, apparently, only for Eileen.

"What're y' all doin' in th' kitchen?" Toby asked, now sticking his head in the doorway, his father right behind. Father and son had both been hitting the Christmas cheer rather heavily, and both had reached the 'jolly' stage.

"We're experimenting with 'Leen's ability to read your son's mind," said Wensley. "It seems to improve communication. Imagine just thinking what you want without having to say anything."

"Wait a mo'," said Toby. "Is 'at why he's slow? 'Cause 'Leen's doin' somewhat to his mind?"

The three women and Wensley were taken aback. "Your son," said Constantina firmly, "is not slow."

"Ted Heseltine says Polly says Edith Philips says 'e might be slow 'cause 'e ain't talkin' yet."

"That's rubbish. No baby his age is talking yet."

Toby was beginning to get steamed, and made an effort to enunciate clearly. "You know wha' I mean. I mean he ain't talkin' baby talk. I ain't heard a wa-wa or a goo-goo out o' him in his entire life. He's nigh a year old. How come he ain't prattling? There's some beginning to think he's slow."

"He's not prattling, as you call it," Constantina retorted, "because he's smart enough to know he doesn't have to. Why learn to talk when all you have to do is think about what you want and you get it?"

"Well then she's got t' stop doin' it. She's got t' make him 'ave t' talk so 's the neighbors don't get the idea he's slow. Once they start thinkin' y're slow, they've got y' pegged for the rest o' yer life."

"Now Toby, don't be so harsh!" Eileen cried.

"The lad may have a point," said Wensley.

"You'd better be awfully sure of yourselves before you go messing in my grandson's head{\el}"

"Nobody's gonna call my son slow!"

As the exchange heated toward argument level, no one noticed that the child in the corner was watching and listening intently, the place behind his eyes sealed shut, guarded and wary. He didn't understand the words, or what the argument was about, but he knew that the people in the room were angry, and it had something to do with him.

Nora made them stop before it went too far. "For crying out loud, it's Christmas Eve! Toby, you take the chicken, Ned the potatoes, Dad Snape the peas. It's time for supper!"

"But we got to{\el}"

"Toby! Not one more word. Tomorrow. We'll discuss it tomorrow when we're calmer." She thrust the platter with the roast chicken into his hands, turned him around, and pushed him into the sitting room where they'd set up card tables for their Christmas Eve feast. Wensley'd brought a bottle of wine for Toby to open. The plum pudding and eggnog were for later.

There was even a little plate with his special favorites for Russ, who could not yet eat everything they were eating. Eileen held out a hand to him and, since the adults were no longer arguing, he solemnly got to his feet, grasped her outstretched finger in a small fist, and let her lead him into the sitting room. That was when he saw the Christmas tree.

As the rest busied themselves setting the tables, Russ, his eyes wide and wonder-filled, crossed the room on his short little legs and reached out a hand to take one of the pretty, sparkling ornaments. "No, Russ. Don't touch," Eileen called to him, and he put the hand down. She turned back to the table, confident that he would obey.

He did obey. She said he must not touch. Russ again held out his right hand to within two inches of the ornament. Slowly, gently, the gaudy ball of blue and gold moved, arcing lazily outward on its hook as if drawn by a magnet, until it reached the waiting fingers. He had not touched the ornament, the ornament had touched him.

The only one in the room to notice was Wensley, who held his breath as he watched the little pointed face with the glittering dark eyes and soft black hair concentrate on the fulfillment of its desire. \emph{Slow, Toby? I don't think so. There's a brain inside that head, whether he talks or not, and whatever else they may think, no one's ever going to think he's slow.}

The next morning, after a fine breakfast, the seven gathered around the Christmas tree to open presents. These were mostly small practical things—a warm pair of gloves, a new cap—because they were poor working-class people who had to take care of each penny, shilling, and half crown. (Not farthings, of course. Eileen was going through every pocket and drawer in the house to find all the farthings, which she would spend in her shopping during the coming week, for with the new year they would become worthless.)

The only one in the house who got frivolous presents in addition to the more practical clothing was Russ. He opened his own gifts, with a little help from Eileen, and was soon playing on the carpet with a toy car that his grandfather had given him. He was the subject of midmorning discussion.

"Much as I hate to admit it, 'Leen," said Constantina as she helped pour the tea, "Toby may be right. If the boy isn't making any attempt to communicate with other people because he can communicate so well with you, then maybe he needs a little push."

"But Mum, he's so young. He won't understand it's for his own good. There's never a day in his life when he ain't connected with me. Can I just take that away?"

"There, dear, it isn't really so bad as that," Nora soothed. "It's not like you don't talk t' him every day, too. Just keep talking t' him. He won't lose that. Talk t' him and cuddle him{\el}"

"Not too much," Toby butted in. "No son of mine's gonna be a mollycoddle."

"Be quiet, Toby," said Wensley. "The child's not a year old yet. This is women's business."

"And remember, 'Leen," her mother added, "we only know that you can read Russ. We don't know if Russ can read you. He listens when you talk and doesn't need eye contact to follow your instructions. You're not removing yourself from him, only the crutch."

"Yes, but he's so little{\el}"

"Sometimes," Wensley said, "you have to be cruel to be kind."

They started that afternoon, after the older Snapes had left. Constantina insisted on staying a few days to help Eileen get through the worst of it, and for once Toby let her have her way because in this she was supporting him.

The battle started at three o'clock when Russ toddled into the kitchen and pulled at his mother's apron. She looked down, smiling but avoiding his eyes, and said, "Wha' cha want, dear?"

"Keep it simple," Constantina warned. "Any way he can show you that doesn't involve reading."

Russ continued to tug at the apron, clearly puzzled that he couldn't make his needs understood. Eileen decided to give him a choice. She patted his nappies. "D' ya need changing?" A quick check showed he didn't. She brought two little bottles, one of water and one of juice. "Are you thirsty, Russ? Show mum what you want."

None of it worked. Eileen was bending down closer, trying to prompt, when Russ suddenly grabbed her hair and pulled. Hard.

"Ow! Hey! Lay off, now you{\el}" Eileen cried.

"What's he doing?" Toby was at the kitchen door watching.

"He's trying to pull my head around so I'll look at him."

Toby grinned in spite of himself. "Knows what he wants and not afraid t' try t' get it, eh?" He turned to Constantina. "He ain't really slow, is he?"

"Toby Snape, your son took his first steps when most babies are still crawling. I think he's going to fight for what he wants right now because he doesn't want to give up the easy life. My grandson isn't slow."

"That's all right then," said Toby, and went back into the sitting room.

Meanwhile, Eileen picked Russ up, but he kept reaching for her hair and eyes. For the first time, she and her son had a true difference of opinion, and for the first time it really struck her how odd it was to have a baby who made no noise except for grunts and coos without meaning. Right now, the most natural thing would be for him to be squalling, but he wasn't.

After a while Russ buried his face against Eileen's arm and lay in her lap, rigid and resentful, while Eileen rocked him, calling him a good boy and asking him to show her what he wanted because if he could do that, she would get it for him. Just show her what it was.

The battle went on for days. Russ clung to Eileen's skirts, reaching up to her, or let himself be carried around the house while he pulled at her hair and nose, and stuck his fingers in her eyes. Other times he lay on his stomach on the sitting room carpet, unmoving and unresponsive, a pathetic, lost little figure who couldn't understand what was happening to his once secure world.

\subsection{Monday, January 9, 1961 (nine hours before the last quarter)}

They didn't celebrate Russ's first birthday. It was too soon after Christmas for any extra expense, Russ did not understand birthdays in any case, and the boy was still being withdrawn and resentful.

"Is 'e bein' sullen?" Toby demanded when he arrived home from the mill by way of the local. He was later than usual, and gin was clearly the reason why. "A boy's s'posed to be a comfort n' support for 'is dad. Ain't s'posed to be sullen."

"Don't be hard on him, Toby. He doesn't understand."

"Some 'un should make 'im understand. Where's m' supper?"

Supper was a source of tension, too. "Why don't 'e eat? A man works 'ard for th' food on 's table and th' sullen witch brat don't eat."

"He hasn't eaten anything all day. I don't know if it's 'cause he's upset or he's sick."

"You don't go t' no bleeding doctor! Bloody National Health can't put a doctor in a man's town, 'e pays taxes and still 'as t' pay a bleeding doctor!"

"What if he's sick?"

"Ain't that wha' cher mum does? Wha' good's a witch, she can't physic 'er own kin?"

After supper, Toby pulled a partial bottle of gin from a cupboard, a bottle left over from Christmas, and after he'd poured a drink or two the problem became worse, but clearer. "That bleedin' horse's behind Evans come down from 'is la-di-da office t'day t' tell us we ain't com-pe-ti-tive. We got to ee-co-no-mize, or the mill's closin'. Askin' us t' do same work in shorter hours. 'T ain't right. A man works 'is whole life 'til he gets where 'e can afford a home 'n a wife 'n family, 'n they ups and takes it away from 'im." He poured another glass and downed it in a gulp.

Eileen froze. "You ain't redundant, are you, Toby?" She was trying to think of a way to take the bottle from him.

"What's th' difference? Less hours, less pay. Bleeding managers ain't takin' less pay, I'll wager!"

"We can still make do, Toby. I can clean and sew{\el}"

"Man's s'posed t' provide for 'is wife! 'E can't do that, 'e ain't a man!" Another drink.

"You're a man, Toby, and a good one. 'T ain't your fault the mill's on hard times."

"Wha's a man t' do, 'Leen? Got a wife 'n kid t' take care of{\el}" Toby looked across the room to where Russ was sitting on the floor, toys abandoned, quiet and resentful. "Com'ere son," Toby called to him. "Come t' yer dad."

Russ didn't move. Worse, he turned his head away. "It's all right," Eileen said quickly. "I can put him to bed in the small room. It'll be just you and me."

"No, I want m' son. Get over here, boy, 'n comfort yer dad." When Russ still didn't move, Toby rose, Eileen trying to restrain him, and stomped over to the boy. He bent down to take Russ's hand, saying, "Y' come to yer dad now," but Russ pulled the hand away and shrank from his father.

"Y' ungrateful little brat!" Toby roared. "I'll teach you what for!" He seized Russ and lifted him, holding him tightly while the toddler wriggled and squirmed and pushed with tiny fists.

"Toby! He's just a baby!" Eileen screamed, trying to break his grip and pull the child away. "Leave him be! Give him to me!"

"Shut yer gob, woman! E's gonna sit with 'is father like a proper son, and not a sullen witch's brat!" Toby yelled back. "'Ere I thought you was a wife, 'n you been teachin' 'im 'gainst me all this time!"

Russ was kicking now, twisting and squirming as his parents shouted at each other, his face reddening and his fists flailing. Suddenly he, too, was screaming—howling and wailing with infant rage and fear. The sound was so shocking to Toby that he staggered back against the little table next to the sofa, tipping over his glass, and released the child to Eileen.

"Wha's 'at?" Toby stammered.

"It's your son," Eileen replied. "He's crying."

"Thought 'e didn't cry."

"He does now." She looked deeply into the dark eyes, saw the need, and carried Russ up to his crib where she crooned to him and settled him down with his teddy bear and his favorite soft cloth, then went back down to comfort the stricken Toby, who had to face the cold outside world alone. Toby was staring at the floor by the cupboard where the shattered gin bottle lay. "Didn't know I knocked into it," he said by way of an apology as she cleaned up the mess.

After that, Russ had no trouble making his needs known. The next morning he clambered into Eileen's lap and patted her mouth with his hand, opening and closing his own mouth in a pantomime of speech. She started speaking baby-talk to him, and he watched her lips intently, mimicking their movements. Two days later he said, "Mama," and Toby was in transports of joy.

"You're going to regret wanting him to start talking," Eileen told Toby the following week, and it was true they could no longer get the boy to be quiet. He babbled and prattled and talked nonsense on his cardboard telephone the way he saw people talk on the public phone in the market, and on the day he mastered the sound `g,' he went around the house crowing, "Ga-ga-ga-ga," all day long. He said `da-da' and `wa-da' and `tey' (which meant teddy), and `pu' when his nappies needed changing. Toby had to agree that he was not slow.

The neighborhood was changing. Most of the men at the mill had their hours cut, and it was hard to make do on less than eight pounds a week. Ted and Polly Heseltine had enough saved that they were able to move to Manchester where he could find work. Most of the other wives started walking long, dusty miles to other villages and towns looking for chores to supplement income.

Toby took up an old refrain. "Why can't ya magic us up something, 'Leen? What good's it being married to a witch if she can't help with a coupla pounds here and there? Y' give me the expense of a baby and then don't help out. 'T ain't fair to a man."

With Polly gone, Kate Hanson suddenly became available to watch Russ while Eileen went seeking day employ, and luckily she asked only a meal in return. They decided it would be better if Kate came to the Snape house so that Russ would still be in familiar surroundings. The first day she had to leave him, Eileen was edgy.

"You know Mrs.~Hanson," she told Russ as she crouched down at his level and straightened his smock. "You be a good boy and don't give her any trouble. I'll be back for supper."

"Don't you fret, Eileen," Kate said. "I may not have had any of my own, but I've taken care of all three of Polly's. We'll get along fine."

How fine, Eileen found out late that afternoon when she returned. Mrs.~Hanson was sitting by herself in the front room knitting.

"Where's Russ?" Eileen asked.

"He's been hiding," Kate said calmly. "Practically the moment you left, he crawled into one of the lower kitchen cabinets and hasn't been out all day. He's punishing you for leaving him. It's normal, believe me. It took Georgy a week before he'd come out of the upstairs wardrobe, and I'm his aunt."

"I got to let him know I'm home."

"I'm sure he knows. He'd have heard the door open. Now, Eileen, he's not going t' come rushing into your arms. He's going t' punish you. He'll retreat, and push you away, and scream like a banshee, but that's because he's been saving up all day just t' let you know how unhappy he is. Let him get it all out, and stay calm."

Mrs.~Hanson was right. Eileen couldn't let the boy stay in the cabinet with Toby due home in an hour, so she pulled him out, and he let out a wail that must have been heard clear down the street. He screamed, and kicked, and fought his way off her lap, and tried to get back into the cabinet, and it was all she could do to stay patient, for she was tired, too, so Kate came into the kitchen, and Russ went to her instead of his mother, and Eileen was finally able to fix supper.

Russ continued to punish Eileen after Toby came home. He did this by pushing away from her and snuggling up to his father on the sofa. "What's this?" Toby asked, clearly pleased. "You two 'ad a tiff or something?"

"He's sore at me for leaving him today," Eileen said. "We're not talking."

"Smart boy," Toby chuckled. "Women 're fickle. Y' got t' stick with your mates."

It was only a matter of time, of course, before Kate and Russ reached a \emph{modus vivendi}, and he began to accept her arrival as normal, and to run to greet his mother when she got home. Kate did express some concerns, though.

"Did you ever notice how distant he is from everyone," she said one day. "Like he's outside watching, but never wants to get close?"

"No, I can't say as I have," Eileen replied. "Did something happen today?"

"No, not really. Maybe it's just because Polly's children were different, more outgoing. Never the same for two minutes. They'd be giggling, and then crying, and then so rapt in something they'd never hear you call, and then fighting, and then loving. Russ, except when he's hiding in cupboards, well he's always so{\el} detached."

"He's always been a quiet child," said Eileen, putting on her apron and starting to prepare supper.

Several months later, Kate greeted Eileen at the door with an apologetic air. "I don't know how it happened, 'Leen, but Russ got out the back door and out onto the moor before I noticed the door was open. I was sure that door was latched."

Since Russ was now sitting on the kitchen floor playing with his toy car, Eileen calmly removed her coat. "I see you caught him."

"When it comes t' it," Kate laughed, "I'm faster than he is."

After Kate had left, Eileen sat herself down next to Russ. "You went out by yourself today."

"I went for walk," Russ replied, not looking at her.

"How'd you open the door?"

Russ thought for a moment. "Didn't. Door just opened."

"Why'd it do that?"

The little boy grinned. "I said please."

Eileen sighed, thankful that childish magic set off no alarms. "Russ, you must never go out of the house alone. Always stay with Mrs.~Hanson. If the door opens again, run and tell her. Do you understand?"

"Yes, Mum." And as always, having been given a direct order, Russ obeyed.

Things went on like this for another couple of years, and then in 1964 the mill finally closed. Men like Derrick Philips and Harry Evans got work in Colne and managed to stay in the town while they commuted. Others, like the Catlows and Garnetts, moved to Manchester and Birmingham. Toby got a job in the mine in the next town and came home drunk more often. Things, which had been bad, gradually got worse.
